%%%%%%%%%%%%%%%%%%%%%%% file template.tex %%%%%%%%%%%%%%%%%%%%%%%%%
%
% This is a template file for Web of Conferences Journal
%
% Copy it to a new file with a new name and use it as the basis
% for your article
%
%%%%%%%%%%%%%%%%%%%%%%%%%% EDP Science %%%%%%%%%%%%%%%%%%%%%%%%%%%%
%
%%%\documentclass[option comma separated list]{webofc}
%%%Three important options:
%%% "epj" for EPJ Web of Conferences Journal
%%% "bio" for BIO Web of Conferences Journal
%%% "mat" for MATEC Web of Conferences Journal
%%% "itm" for ITM Web of Conferences Journal
%%% "e3s" for E3S Web of Conferences Journal
%%% "shs" for SHS Web of Conferences Journal
%%% "twocolumn" for typesetting an article in two columns format (default one column)
\documentclass[epj,twocolumn]{webofc}
\usepackage[varg]{txfonts}   % Web of Conferences font
%
% Put here some packages required or/and some personnal commands
%
% Important: please activate and fill the "wocname" command with the exact title of the series for conferences not included in any of the series listed on the top
\usepackage{siunitx}

% Very important: please fill the "woctitle" command with the exact title of the conference
%
\woctitle{Powders \& Grains 2017}
%
%
\begin{document}
%
\title{Collapse of tall granular columns in fluid}
%
% subtitle is optionnal
%
%%%\subtitle{Do you have a subtitle?\\ If so, write it here}

\author{\firstname{Krishna} \lastname{Kumar}\inst{1}\fnsep\thanks{\email{kks32@cam.ac.uk}} \and
        \firstname{Kenichi} \lastname{Soga}\inst{2}\fnsep\thanks{\email{soga@berkeley.edu}} \and
        \firstname{Jean-Yves} \lastname{Delenne}\inst{3}\fnsep\thanks{\email{}} \and
        \firstname{Farhang} \lastname{Radjai}\inst{4}\fnsep\thanks{\email{}}
        % etc.
}

\institute{Department of Engineering, University of Cambridge, UK 
\and
           University of California, Berkeley, US
\and
           INRA, France
\and 
           CNRS, France
          }

\abstract{%
  Avalanches, landslides, and debris flows are geophysical hazards, which involve rapid mass movement
  of granular solids, water, and air as a single phase system. In order to describe the mechanism of
  immersed granular flows, it is important to consider both the dynamics of the solid phase and the
  role of the ambient fluid~\cite{Denlinger2001}. The dynamics of the solid phase alone are
  insufficient to describe the mechanism of granular flows in fluid. It is important to consider the
  effect of hydrodynamic forces that reduce the weight of the solids inducing a transition from
  dense-compacted to dense-suspended flows, and the drag interactions which counteract the movement
  of the solids~\cite{Meruane2010}. Transient regimes characterised by a change in the solid
  fraction, dilation at the onset of flow and the development of excess pore-pressure, result in
  altering the balance between the stress carried by the fluid and that carried by the grains, thereby
  changing the overall behaviour of the flow. In the present study, a two-dimensional coupled Lattice
  Boltzmann LBM – DEM technique is developed to understand the micro-scale rheology of granular
  flows in fluid.

  In the present study, the collapse of a granular column in fluid is studied using 2D LBM - DEM.
  The effect of initial aspect ratio on the run-out behaviour is investigated. The flow kinematics are
  compared with the dry and buoyant granular collapse to understand the influence of hydrodynamic forces
  and lubrication on the run-out. In the case of tall columns, the amount of material destabilised above
  the failure plane is larger than that of short columns. Hence in tall columns, the surface area of the
  mobilised mass that interacts with the surrounding fluid is significantly higher than the short columns.
  This increase in the area of soil - fluid interaction results in an increase in the formation of
  turbulent vortices that alter the deposit morphology during the collapse. It is observed that the
  vortices result in formation of heaps that significantly affect the distribution of mass in the flow.
  ~\cite{Staron2007a} observed that the distribution of mass in a granular flow plays a crucial role
  in the flow kinematics. In order to understand the behaviour of tall columns, the run-out behaviour of a
  dense granular column with an initial aspect ratio of 6 is studied. The collapse of a tall granular
  column on slopes of 0, 2.5, 5 and 7.5 are studied. The permeability of the granular mass was varied
  to understand the effect of permeability on the run-out behaviour. }
%
\maketitle
%
\section{Introduction}
The collapse of a granular column on a horizontal surface is a simple case of granular flow, however a
proper model that describes the flow dynamics is still lacking. Experimental investigations have shown that
the flow duration, the spreading velocity, the final extent of the deposit, and the energy dissipation can be scaled in a quantitative way independent of substrate properties, grain size, density, and shape of the granular
material and released mass~\cite{Lajeunesse2005, Lube2005}. Simple mathematical models based on conservation of horizontal momentum capture the scaling laws of the final deposit, but fail to describe the initial transition regime. Granular flow is modelled as a frictional dissipation process in continuum mechanics but the lack of influence of inter-particle friction on the energy dissipation and spreading dynamics~\cite{Lube2005} is surprising.

\section*{Acknowledgements}
The author would like to thank the Cambridge Commonwealth and Overseas Trust for the financial support to the
first author to pursue this research.

%
% BibTeX or Biber users please use (the style is already called in the class, ensure 
% that the "woc.bst" style is in your local directory)
% \bibliography{name or your bibliography database}
%
% Non-BibTeX users please use
%
\bibliography{references} 

\end{document}

% end of file template.tex
